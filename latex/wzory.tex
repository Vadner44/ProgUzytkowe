\documentclass[a4paper,12pt]{article}
\usepackage[MeX]{polski}
\usepackage[utf8]{inputenc}
\usepackage{amsthm}
\usepackage{algorithm}
\usepackage{algorithmic}
\floatname{algorithm}{Algorytm}
\renewcommand{\algorithmicrequire}
{\textbf{Założenia wstępne:}}
\renewcommand{\algorithmicensure}
{\textbf{Na wyjściu:}}
\renewcommand{\algorithmicwhile}
{\textbf{Dopóki}}
\renewcommand{\algorithmicdo}
{\textbf{wykonuj}}
\renewcommand{\algorithmicif}
{\textbf{Jeśli}}
\renewcommand{\algorithmicthen}
{\textbf{wtedy}}
\renewcommand{\algorithmicelse}
{\textbf{Inaczej}}


\title{Wzory matematczne}
\author{Patryk Kowalewski}


\begin{document}

\maketitle

\begin{abstract}
Kilka wzorów matematycznych użytych w \LaTeX.
\end{abstract}

\section{Twierdzenie Pitagorasa}
\newtheorem*{twr}{Twierdzenie}
\theoremstyle{plain}
\begin{twr}[Pitagorasa] 
W dowolnym trójkącie prostokątnym suma kwadratów długości przyprostokątnych jest równa kwadratowi długości przeciwprostokątnej tego trójkąta. 
Zgodnie z oznaczeniami na rysunku obok zachodzi tożsamość:
$$a^{2}+b^{2}=c^{2}$$
Geometrycznie oznacza to, że jeżeli na bokach trójkąta prostokątnego zbudujemy kwadraty, 
to suma pól kwadratów zbudowanych na przyprostokątnych tego trójkąta będzie równa polu kwadratu zbudowanego na przeciwprostokątnej.

\section{Inne Wzory}

$$ \lim_{x \to 0} \frac{\sin x}{\tan x}=1$$

$$\lim_{n \to \infty}\sum_{k=1}^n \frac{1}{k^2}= \frac{\pi^2}{6}$$

\begin{displaymath}
sign(x) = \left\{ \begin{array}{ll}
1, & \textrm{jeśli $x>0$}\\
0, & \textrm{jeśli $x=0$}\\
-1, & \textrm{jeśli $x<1$}\\
\end{array} \right.
\end{displaymath}

$$a_{1} x^{2} e^{-\alpha t}
a^{3}_{ij} e^{x^2} \neq {e^x}^2$$

\section{Algorytm}

\begin{algorithm}
\caption{Obliczenie $x^n$}
\begin{algorithmic}[2]
\REQUIRE $n\ge 0$
\ENSURE $a=x^n$
\STATE $k\leftarrow n$; $a\leftarrow 1$; $b\leftarrow x$;
\WHILE[Niezmiennik: $x^n=a\cdot b^k$]{$k>0$}
\IF{$k$ jest liczbą parzystą}
\STATE $k\leftarrow k/2$;
\STATE $b\leftarrow b\cdot b$;
\ELSE[$k$ jest liczbą nieparzystą]
\STATE $k\leftarrow k-1$;
\STATE $a\leftarrow a\cdot b$;
\ENDIF
\ENDWHILE
\end{algorithmic}
\end{algorithm}


\end{document}